%% \typeout{IJCAI--22 Instructions for Authors}

% These are the instructions for authors for IJCAI-22.

\documentclass{article}
\pdfpagewidth=8.5in
\pdfpageheight=11in
% The file ijcai22.sty is NOT the same as previous years'
\usepackage{ijcai22}

% Use the postscript times font!
\usepackage{times}
\usepackage{soul}
\usepackage{url}
\usepackage[hidelinks]{hyperref}
\usepackage[utf8]{inputenc}
\usepackage[small]{caption}
\usepackage{graphicx}
\usepackage{amsmath}
\usepackage{amsthm}
\usepackage{booktabs}
\usepackage{algorithm}
\usepackage{algorithmic}
\usepackage{multirow}
\urlstyle{same}

% the following package is optional:
%\usepackage{latexsym}

% See https://www.overleaf.com/learn/latex/theorems_and_proofs
% for a nice explanation of how to define new theorems, but keep
% in mind that the amsthm package is already included in this
% template and that you must *not* alter the styling.
\newtheorem{example}{Example}
\newtheorem{theorem}{Theorem}

% Following comment is from ijcai97-submit.tex:
% The preparation of these files was supported by Schlumberger Palo Alto
% Research, AT\&T Bell Laboratories, and Morgan Kaufmann Publishers.
% Shirley Jowell, of Morgan Kaufmann Publishers, and Peter F.
% Patel-Schneider, of AT\&T Bell Laboratories collaborated on their
% preparation.

% These instructions can be modified and used in other conferences as long
% as credit to the authors and supporting agencies is retained, this notice
% is not changed, and further modification or reuse is not restricted.
% Neither Shirley Jowell nor Peter F. Patel-Schneider can be listed as
% contacts for providing assistance without their prior permission.

% To use for other conferences, change references to files and the
% conference appropriate and use other authors, contacts, publishers, and
% organizations.
% Also change the deadline and address for returning papers and the length and
% page charge instructions.
% Put where the files are available in the appropriate places.

% PDF Info Is REQUIRED.
% Please **do not** include Title and Author information
\pdfinfo{
/TemplateVersion (IJCAI.2022.0)
}

\title{IJCAI--22 Formatting Instructions}

% Single author syntax
%% \author{
%%     Author Name
%%     \affiliations
%%     Affiliation
%%     \emails
%%     pcchair@ijcai-22.org
%% }

% Multiple author syntax (remove the single-author syntax above and the \iffalse ... \fi here)

\author{
Arumoy Shome$^1$
\and
Lu{\'\i}s Cruz$^1$\And
Arie van Deursen$^{1}$
\affiliations
$^1$Delft University of Technology\\
\emails
\{a.shome, l.cruz, arie.vandeursen\}@tudelft.nl
}


\begin{document}

\maketitle

\begin{abstract}
%% TODO
\end{abstract}

\appendix

\section{Introduction}\label{sec:intro}
%% TODO

%% present a general motivation of the problem space & the problem we
%% are trying to solve

%% present research questions that we want to answer along with brief
%% preview of results

%% RQ1: can we minimise fairness testing by only testing the data?
%% RQ2: can we minimise to data fairness testing when experimenting
%% with training size?
%% RQ3: can we minimise to data fairness metrics when experimenting
%% with feature size?

\section{Related Work}\label{sec:related}
%% TODO

%% summarise the current work on fairness testing; touch upon this
%% aspect from a SE perspective; the fairness metrics we have (and why
%% we use specific ones);

\section{Methodology}\label{sec:method}
%% TODO should we include numbers conditioned on
%% privileged/unprivileged?

\begin{table}[htb]
  \centering
  \caption{Datasets used in the study}
  \begin{tabular}{p{0.3\linewidth} p{0.1\linewidth} p{0.1\linewidth} r r r}
    \toprule

    \textbf{Name} & \textbf{Protected} & \textbf{Abbr.} &
    \textbf{Total} & \textbf{Positive} & \textbf{Negative}\\

    \midrule

    \multirow{2}{*}{Adult Income \cite{CITEME}} & sex & adult-sex & 45222 & 11208 &
    34014\\
      & race & adult-race & 45222 & 11208 & 34014\\
    \multirow{2}{*}{Compas Score \cite{CITEME}} & sex & compas-sex &
    6167 & 3358 & 2809\\
      & race & compas-race & 6167 & 3358 & 2809\\
    Bank Marketing \cite{CITEME} & age & bank-age & 30488 & 3859 &
    26629\\
    \multirow{2}{*}{German Credit \cite{CITEME}} & sex & german-sex &
    1000 & 700 & 300\\
      & age & german-age & 1000 & 700 & 300\\
    Medical Survey 2021 \cite{CITEME} & race & meps-race & 15675 &
    2628 & 13047\\
    \bottomrule
  \end{tabular}
  \label{tab:datasets}
  \end{table}
%% TODO brief commentary on tab:datasets, the ones we use (and why),
%% the range of examples; short comment on pre-existing bias in the
%% dataset.

%% list the fairness metrics we use in our analysis & why
\begin{equation}
  DI_{data} = \frac{P(Y=1|D=0)}{P(Y=1|D=1)}
  \label{eq:di-data}
\end{equation}

\begin{equation}
  DI_{model} = \frac{P(\hat{Y}=1|D=0)}{P(\hat{Y}=1|D=1)}
  \label{eq:di-model}
\end{equation}

\begin{equation}
  SPD_{data} = P(Y=1|D=0)-P(Y=1|D=1)
  \label{eq:spd-data}
\end{equation}

\begin{equation}
  DI_{model} = P(\hat{Y}=1|D=0)-P(\hat{Y}=1|D=1)
  \label{eq:spd-model}
\end{equation}

%% list the ML models we use in our analysis & why

%% outline the data collection setup; break this down by the
%% experiments (we have two); explain the slicing mechanism in each
%% experiment, how & why we are shuffling the example & feature orders

%% list the fairness metrics we collect; how we aggregate them, etc.

%% TODO definitely need to diagram; hard to explain experimental
%% design using only words
Figure \ref{fig:exp-training-sets} presents the experimental design
for the training sets experiment. The dataset is split into a training
(75\%) and testing (25\%) subset. Random samples of varying size
between 10\% and 100\% are then taken to train the machine learning
models. The trained models are used to derive predictions from the
test subset. The fairness metrics are calculated twice: once using the
random training sample (data fairness metrics) and again using the
predictions derived from the models trained on the corresponding
random sample.

\section{Results \& Discussion}\label{sec:results}
%% list & explain the two tests we use to evaluate relationship
%% between data & model metrics (we have two: correlation & linear
%% regression)

%% present the results from the two experiments & our analysis

%% exp-training-sets: emphasise the importance of data quality. If we look at the underlying data, the 
%% we need to link the results & our discussion to the research
%% questions we are trying to answer

\section{Threats to Validity}\label{sec:threats}
%% TODO

\section{Conclusion}\label{sec:conclude}
%% TODO

\bibliographystyle{named}
\bibliography{report}

\end{document}

